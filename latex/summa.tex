\documentclass{summa}

\title{Theologische Summa van den H. Thomas van Aquino}
\author{Door een groep Dominicanen}
\date{1927}

\begin{document}

\maketitle
\setcounter{tocdepth}{1}
\tableofcontents

% ============ CHAPTER FORMAT ============

\begin{quaestio}{Over de gewijde leer}{WAT ZIJ IS EN WAT ZIJ BEVAT}

  % ============ SECTION FORMAT ============

  \begin{articulus}{Is er buiten de wijsbegeerte nog een andere leer nodig?}

    % ============ SUBSECTION FORMAT ============

    \begin{objectiones}
    \item { Men beweert, dat er buiten de wijsbegeerte geen andere leer nodig is. De mens behoeft immers niet te streven naar wat boven zijn verstand is, volgens het woord van de Prediker (3, 22) : “Zoek niet naar wat boven U verheven is”. Welnu, wat onder het bereik van de rede valt, wordt voldoende verhandeld in de wijsbegeerte. Een andere leer buiten de wijsbegeerte is dus overbodig. }
    \item { Iedere leer handelt, alleen over het zijnde. Inderdaad, alleen het ware, — dat met het zijnde gelijkstaat, — is het voorwerp van de wetenschap. Welnu de wijsbegeerte handelt over al het zijnde, ook over God. Er is immers een deel der wijsbegeerte dat Godgeleerdheid of goddelijke wetenschap genoemd wordt, naar de leer van de Wijsgeer in het VIe Boek der Metaphysica (Ve B., 1e H., Nr 7). Het is dus niet nodig, dat er buiten de wijsbegeerte nog een andere leer is. }
    \end{objectiones}
    
    \begin {contra}
      Daartegenover staat echter wat we lezen in de Ie Brief aan Timotheüs (3, 16) : “Alle door God ingegeven Schriften hebben hun nut om te onderrichten, om te weerleggen, om te berispen, om te onderwijzen tot gerechtigheid”. Welnu, de H. Schrift behoort niet tot de wijsbegeerte, want ze is door God ingegeven, terwijl de wijsbegeerte door de menselijke rede is uitgewerkt. Het is dus nuttig, dat er buiten de wijsbegeerte een andere wetenschap is, door God ingegeven.
    \end {contra}
    
    \begin {respondeo}
      Buiten de wijsbegeerte, die een werk is der menselijke rede, is er voor het heil der mensen een leer nodig, die steunt op de goddelijke openbaring. De mens is immers naar God gericht als naar een doel dat ’s mensen begrip te boven gaat, volgens het woord van Isaïas (64, 4) “Zonder U, God, heeft geen oog gezien wat Gij voorbereid hebt voor hen die U beminnen”. Welnu, de mensen moeten het doel te voren kennen, want zij moeten er hun inzichten en handelingen naar richten. Het is dus noodzakelijk voor ’s mensen zaligheid dat sommige waarheden die de rede te boven gaan door goddelijke openbaring meegedeeld worden.  
      
      Wat nu de goddelijke waarheden betreft, welke de menselijke rede kan bereiken, ook deze dienen door God geopenbaard, en dit omdat de waarheden die de rede aangaande God kan achterhalen, niet dan door enkelen, en na langen tijd, en met veel dwalingen vermengd kunnen begrepen worden. En van de kennis van die waarheden hangt nochtans heel ’s mensen heil af, dat in God gelegen is. Opdat nu de mensen algemener en zekerder de zaligheid zouden bereiken, is het nodig, dat ze door goddelijke openbaring over God onderwezen worden.  
      
      Wij mogen dus besluiten, dat buiten de wijsbegeerte, het werk der rede, de gewijde leer, het werk der openbaring, nodig is.
    \end {respondeo}
    
    \begin{responsiones}
    \item { Wat boven de menselijke rede is, moet zij niet trachten te achterhalen, maar wanneer God het openbaart, moet zij het door het geloof aanvaarden. Daarom wordt t.a.pl. ook gezegd : “Veel van wat boven ’s mensen kennis verheven is, werd U geopenbaard”. Daarin nu juist bestaat de gewijde leer. }
    \item { De verscheidenheid der wetenschappen ontstaat uit hun verschillend formeel kenmiddel. Astronomen en natuurkundigen bewijzen immers beiden eenzelfde stelling, b.v. de bolvormigheid der aarde; maar de sterrenkundige doet dit door middel van een wiskundige bewijsvoering die van de stof abstraheert; de natuurkundige, door middel van een waarneming van de stof. Hieruit volgt, dat vraagstukken die, in zover zij in het bereik der menselijke rede liggen, door de wijsbegeerte behandeld worden, bovendien ook het voorwerp van een andere wetenschap kunnen zijn, welke ze kent door het licht der goddelijke openbaring. De godgeleerdheid welke bij de gewijde leer behoort, verschilt dus soortelijk van die godgeleerdheid, welke een deel der wijsbegeerte is. }
    \end{responsiones}
  \end{articulus}

  %% ========================

  \begin{articulus}{Is er buiten de wijsbegeerte nog een andere leer nodig?}
    \begin{objectiones}
    \item { Men beweert, dat de gewijde leer geen wetenschap is. Een wetenschap immers ontstaat uit door zich zelf klaarblijkelijke beginselen. De gewijde leer integendeel ontstaat uit de geloofsartikelen, die niet uit zich zelf klaarblijkelijk zijn, want dan zouden ze door iedereen aangenomen worden. Maar “niet iedereen is gelovig”, zegt de Apostel (IIIe Brief aan de Thessalonicenzen, 3, 2). Dus is de gewijde leer geen wetenschap. }
    \item { Er is geen wetenschap van het individuele. Welnu, de gewijde leer handelt over individuele dingen, b.v. over de geschiedenis van Abraham, Isaac en Jacob. Dus is de gewijde leer geen wetenschap. }
    \end{objectiones}
    
    
    \begin {contra}
      Dit is echter strijdig met wat Augustinus zegt in zijn werk Over de Drie-eenheid (XIVe B., Ie H.) : “Tot de gewijde leer wordt alleen datgene gerekend, waardoor het allerheilzaamste geloof ontstaat, waardoor het ontwikkeld, verdedigd en versterkt wordt.”
    \end {contra}
    
    \begin {respondeo}
      De gewijde leer is een wetenschap. Er dient opgemerkt, dat er tweeërlei wetenschappen zijn : sommige ontstaan uit beginselen die gekend worden door het natuurlijk licht van het verstand, zoals de rekenkunde, de meetkunde en dergelijke; andere ontstaan uit beginselen die gekend worden door een hogere wetenschap, zoals de wetenschap van het perspectief voortvloeit uit de beginselen der rekenkunde. De gewijde leer nu is een wetenschap in deze tweede betekenis, want ze vloeit voort uit beginselen, die door een hogere wetenschap gekend worden, nl. door de wetenschap van God en die van de heiligen. Evenals de muziek de beginselen aanneemt die door de rekenkunde gegeven worden, zo neemt ook de gewijde leer de beginselen aan, door God geopenbaard.
    \end {respondeo}
    
    % Responsiones
    \begin{responsiones}
    \item { De beginselen van iedere wetenschap zijn ofwel uit zich zelf klaarblijkelijk, ofwel te herleiden tot een hogere wetenschap. Dit, laatste is het geval met de gewijde leer, zoals in de loop van het artikel gezegd werd. }
    \item { Individuele dingen worden in de gewijde leer niet behandeld als hoofdzaak, maar worden alleen aangevoerd óf als voorbeelden, gelijk men ook in de zedenleer doet, óf om het gezag toe te lichten van hen door wie de goddelijke openbaring tot ons gekomen is. De goddelijke openbaring is immers de grondslag van de H. Schrift of de gewijde leer. }
    \end{responsiones}
  \end{articulus}
\end{quaestio}
\end{document}
